\section{Introduction} % Chapter title

\label{ch:introduction} % For referencing the chapter elsewhere, use \autoref{ch:introduction} 


\subsection{Idée principale}


Nous assistons à un mépris croissant dans notre société capitaliste. Ainsi, même si nous vivons dans un système dit démocratique, nous sommes tous confrontés à un monde très complexe où les plus puissants exercent leur domination sur le reste de la population impuissante. En effet, le pouvoir est de plus en plus sous le contrôle et la domination des oligarques insensibles qui composent notre société. Cette minorité, la classe bourgeoise, a toujours exercé une emprise écrasante sur les masses, la classe ouvrière. Cette domination est souvent sans morale.

De plus, cette même minorité se donne souvent beaucoup de mal pour occuper des pôles importants tels que l'appareil d'État, les entreprises et les partenaires internationaux, et exercer ainsi, en toute discrétion, des exactions et des exploitations sur les pauvres. Ce pouvoir, injustement séquestré, favorise entre autres:
\begin{enumerate}
  \item la mauvaise gouvernance : nos forces de police, nos administrations, nos hôpitaux, etc. sont tous mal gérés,
  \item la corruption à tous les niveaux de l'appareil d'Etat et des autres structures,
  \item un manque total de considération pour la souffrance du peuple,
  \item etc.
\end{enumerate}

De plus, le manque de transparence et la corruption ne se limitent pas aux entreprises et aux organismes publics, mais touchent toutes les fonctions, de l'administration aux systèmes éducatifs (écoles et universités) en passant par les petites et moyennes entreprises. Ainsi, dans cette lutte pour la transparence au détriment de la corruption, de la mauvaise gouvernance, nous devons prendre en compte tous les moyens d'identifier ces phénomènes et de les combattre efficacement. Nous devons nous donner les moyens - financiers, matériels ou autres - de lutter contre la corruption, la mauvaise gouvernance et l'injustice. Bien que ce phénomène soit présent partout, que ce soit dans les pays développés, pauvres ou en voie de développement, seules sa forme et sa nature changent, mais le mal est le même. 


\subsection{Problématique}


Dans toutes les sociétés, on peut parler des abus que les plus pauvres d'entre nous subissent dans leur vie quotidienne. Ces abus peuvent aller du grand mépris que l'on subit sur le lieu de travail, mais aussi dans les lieux administratifs, aux services que l'on paie et que l'on consomme, de l'escroquerie à la fraude, bref n'importe où. De plus, la plupart de ces abus sont encouragés par une mauvaise gouvernance et exaspérés par l'absence de moyens de les dénoncer aisément. Ces comportements nous sont si désagréables, trop souvent, que nous finissons par les assimiler et les considérer comme normaux.

Prenons par exemple le grand journaliste d'investigation Babacar Touré, qui a beaucoup oeuvré en faveur des personnes opprimées, victimes d'injustices souvent incroyables de la part de divers types d'acteurs étatiques ou civils. Sa défense sans faille des victimes de l'affaire Boffa-Boyotte en est un exemple, dont il a assuré le suivi et la documentation de manière exhaustive~\footnote{Justice: Le dossier des tueries de Boffa Bayotte renvoyé devant la chambre d'accusation le 14 octobre.~\href{https://kewoulo.info/justice-dossier-tueries-de-boffa-bayotte-renvoye-devant-chambre-daccusation/}{Justice: Le dossier des tueries de Boffa Bayotte renvoyé devant la chambre d'accusation le 14 octobre}}
\footnote{Boffa Bayottes: Les terribles vérités jamais dites sur l'assassinat des 14 tués de Bourofaye.~\href{https://youtu.be/hodC8_OiDWM?feature=shared}{Boffa Bayottes: Les terribles vérités jamais dites sur l'assassinat des 14 tués de Bourofaye (YouTube)}}. Il a permis à ces victimes de bénéficier d'une couverture médiatique et d'être connues et reconnues de tous par rapport aux atteintes portées à leurs modestes personnes. Plus de détails sont disponibles sur le site du journaliste, Kéwulo, et sur sa chaîne YouTube, ainsi qu'à travers les nombreuses interviews auxquelles il a participé.

Faisons un saut dans l'enseignement supérieur. Outre le fait que toutes les universités sont très en retard, un système très autoritaire et arbitraire s'est installé. Un professeur peut être absent de la plupart de ses cours et oser évaluer ses étudiants sur l'ensemble du programme. Cela pénalise les étudiants courageux à qui l'on n'a rien demandé d'autre que d'apprendre leurs leçons à la manière de Kyle XY (une série télévisée). Heureusement, on peut toujours compter sur ces mêmes professeurs des ajustement de note allant même jusqu'à +10. Sans parler des queues humiliantes que font les étudiants pour le retrait de leurs bourses, ce qui a même donné naissance à une mafia de petits corrompus qui imposent des commissions aux pauvres étudiants en détresse qui ne cherchent qu'à retirer leurs bourses dûment méritées.

En ce qui concerne nos entreprises, je vous renvoie à la dernière conférence de \textit{Guy Marius Sagna} sur les femmes enceintes licenciées arbitrairement et interdites de créer ou d'appartenir à un syndicat, et sur de nombreux employés qui n'ont jamais vu leur contrat ou qui n'ont tout simplement jamais eu de contrat.~\footnote{DIRECT - Conférence de Presse de Guy Marius Sagna.~\href{https://www.youtube.com/live/KQMg7fPc62I}{DIRECT (YouTube) - Conférence de Presse de Guy Marius Sagna}.}. Je ne m'étendrai pas sur l'entreprise UNO à Diamgnadio, qui fait travailler des hommes et des femmes debout pendant des heures, provoquant des avortements, des maux de dos, etc.\newline
Je vais également citer le témoignage de \textit{Borom rew mi TV} concernant l'entreprise \texttt{Uniparco cosmetique} et ses exactions et mauvais traitement vis-à-vis de ses employés.~\footnote{Borom rew mi sur le SCANDAL Uniparco cosmetique.~\href{https://www.youtube.com/watch?v=MEpqNZ72tMk}{Borom rew mi sur le SCANDAL Uniparco cosmetique (YouTube)}.}

Dans nos foyers, de nombreuses femmes sont recrutées pour aider aux tâches ménagères. Cependant, ces femmes subissent toutes sortes de traitements invraisemblables et de manque de considération, tels que des abus physiques, des attouchements sexuels, des viols et même pire. Les horaires et les tâches ne sont jamais clairement définis et sont souvent arbitraires pour un salaire moindre. En conséquence, certaines rentrent très tard, pour revenir très tôt le matin. Les heures de travail dépassent largement la durée légale du travail, qui est fixée entre 40 et 45 heures par semaine~\footnote{Durée du travail au Sénégal.~\href{https://africapaierh.com/juridique/duree-du-travail-au-senegal/}{Durée du travail au Sénégal}.}. Ces femmes restent généralement silencieuses, vivant leur souffrance sans affection ni soutien moral. De plus, elles font l'objet de licenciements arbitraires, souvent sans indemnités. Bref, ces femmes sont livrées à elles-mêmes, le plus souvent avec un salaire destiné à couvrir quelques frais de scolarité ou à subvenir aux besoins de la famille. Elles n'ont pas de voix.

Dois-je continuer ?

Parlons des moyens dont nous disposons pour en parler. La démocratie véritable prône la liberté d'expression, le pouvoir censé appartenir au peuple. Or, cette démocratie, partout où elle est mise en oeuvre, à commencer par le Sénégal (dans les gouvernements précédents), se trouve instrumentalisée au profit des plus riches. Cette instrumentalisation est facilitée par les \textbf{médias}, souvent appelés le quatrième pouvoir. Certains de ces médias sont souvent orientés par les pouvoirs publics et par des lobbies aux moyens considérables, et subventionnés de notre poche~\footnote{L'aide aux médias portée à 700 millions F Cfa.~\href{https://www.senegalservices.sn/actualite/laide-aux-medias-portee-a-700-millions-f-cfa}{L'aide aux médias portée à 700 millions F Cfa}}, pour manipuler et désorienter voire diviser l'opinion publique. Ils sont capables de porter n'importe quel chapeau au gré de leurs intérêts et de ceux qui les financent. Néanmoins, certains sont toujours restés proches des opprimés. Comme Walf TV au Sénégal.~\footnote{\href{https://walf-groupe.com/walf-tv/}{https://walf-groupe.com/walf-tv/}}

D'autres moyens de communication ont également vu le jour avec le développement de l'internet. Des plateformes telles que YouTube, Facebook (aujourd'hui META) et Whatsapp, Telegrams et TikTok existent et ont favorisé l'émergence de plusieurs mouvements activistes qui s'efforcent constamment d'apporter une contribution de qualité à la sensibilisation, la dénonciation et à l'éveil du peuple, tel que Xaalat TV~\footnote{\href{https://www.youtube.com/@Xalaattv}{Chaine YouTube de Xalaat TV}}. Cependant, tous ces outils ne sont pas les nôtres. Deplus, la grande majorité d'entre eux sont entre les mains d'oligarques capitalistes, moins motivés par la liberté d'expression. N'avez-vous pas suivi nos philanthropes qui nous certifient que le mensonge et la manipulation par l'IA constituent la plus grande menace pour l'humanité~\footnote{~\href{https://unherd.com/newsroom/beware-the-wefs-new-misinformation-panic}{Attention à la nouvelle panique du World Economic Forum (WEF) en matière de désinformation}}. En conséquence, Facebook fait déjà du bon travail en limitant sévèrement les contenus politiques des divers activistes qui mettent en lumière les nombreux mensonges des politiciens~\footnote{~\href{https://next.ink/128813/contenus-politiques-que-presage-le-changement-de-politique-de-meta}{Meta a modifié ses conditions d'utilisation pour, peu à peu, cesser de recommander les contenus politiques sur Facebook.}}. Ne soyez pas surpris quand Youtube et Tiktok feront de même. Nous avons besoin d'une solution qui nous appartient, une solution sénégalaise, et potentiellement africaine, en accord avec nos valeurs africaines ancestrales.

De plus, malgré les nombreux moyens mis à notre disposition, il est très difficile de s'y retrouver et il existe peu de moyens de lutter contre la désinformation véhiculée par certains acteurs malveillants. Ils n'offrent pas non plus de statistiques permettant de recouper plus facilement les faits et les sujets abordés. Il est difficile, par exemple, de se faire une idée précise de l'action d'un gouvernement ou d'identifier les entreprises les moins respectueuses des droits de leurs employés. Il peut en résulter un manque d'intérêt de la part du public, ou un manque de prise en compte de l'ampleur d'un phénomène. Nous avons donc besoin d'un outil pour mettre en lumière toutes les décisions politiques, les conditions de travail dans les institutions publiques et privées, etc. Cette lumière peut être une exposition de faits et de décisions tels que des contrats, des abus, etc. pour une visibilité et une transparence totales.

Un autre moyen de lutte est le syndicalisme. Un syndicat est généralement créé pour lutter contre toute oppression visant les personnes qu'il représente, les travailleurs. Cependant, ce dernier n'échappe pas à la corruption de manière à permettre aux dirigeants de percevoir des commissions, les détournant ainsi de leur combat. En France, par exemple, pays plus avancé en matière de protection des travailleurs, on retrouve la CGT épinglée dans des affaires de corruption de ses dirigeants. Le mal qu'elle est censée éradiquer est ainsi aggravé.~\footnote{J'ai été moi-même membre de la CGT pendant 30 ans, avant de rejoindre FO, et j'ai bien vu que la dénonciation n'est jamais facile dans ces organisations: elle entraîne pressions, perte de mandat syndical voire perte de l'emploi dans certains cas!~\href{https://www.lefigaro.fr/vox/societe/2014/10/29/31003-20141029ARTFIG00296-chantage-magouilles-corruption-le-livre-noir-du-syndicalisme-francais.php}{Chantage, magouilles, corruption : le livre noir du syndicalisme français}} Quel est le poids de nos syndicat et la manière dont sont-ils financés ? Si ceux-là même qui sont censés nous défendre des amus abusent [de notre confiance] alors là c'est la meilleure. Je vous laisse imaginer les syndicats dans les pays pauvres.

Il est impossible d'énumérer tous les exemples pour couvrir tous les différents secteurs dans lesquels nous pouvons observer la domination, l'exaction, la corruption et nous trouver impuissants à simplement les dénoncer et ainsi mettre la lumière sur leurs agissements et aider les autorités compétentes à s'engager sur la bonne voie. Nous ne pouvons pas non plus énumérer les outils et les moyens dont nous disposons pour dénoncer librement ces fléaux. Il est donc nécessaire de proposer un moyen efficace et simple, accessible à tous, pour dénoncer toute oppression, quelle qu'elle soit, dans le respect de l'éthique et des valeurs fondamentales qui régissent nos lois et nos moeurs.

\subsection{Organisation du travail}

Aussi, face à toutes ces solutions et à leurs limites, il nous a semblé nécessaire de proposer une solution qui non seulement nous appartiendrait, mais nous permettrait à tous de nous exprimer et de partager nos idées en toute liberté, dans le respect et l'éthique.

Nous pensons qu'il est nécessaire d'avoir nos propres outils, car ils sont assez faciles à développer et à adapter à nos réalités. Lorsque nous vivons dans une maison dont les règles ne correspondent pas à nos valeurs, ou que nos principes et nos idées dérangent les propriétaires, nous la quittons simplement et créons notre propre maison avec nos partenaires, brique par brique.

En effet, les outils existants tels que Facebook ou X (anciennement tweeter) ne sont pas impossibles à créer pour nos propres besoins et réalités. Nous continuons à utiliser les solutions des autres et nous nous plaignons lorsqu'ils modifient leurs règles à leur avantage. Nous avons besoin des nôtres, et nous devons tirer parti de ce que les autres proposent de bon et éviter leurs mauvais côtés.

C'est pourquoi, à travers ce petit projet nommé \ngelmak, nous proposons à notre patrie et avec son aide, puis au monde entier, un outil pour \textbf{dénoncer les exactions} et \textbf{promouvoir les bonnes idées}. En effet, ce projet ne vise pas seulement à promouvoir la dénonciation, mais aussi l'émergence d'acteurs [panafricains] qui participent à l'éveil du peuple, via des panels proposant des idées et des critiques constructives. Ces acteurs peuvent également apporter des précisions et des explications pour une meilleure compréhension des actions menées par nos politiciens, ainsi que des suggestions qui pourraient leur être très utiles. Il ne s'agit donc pas d'un outil contre un gouvernement qui travaille pour et soutenu par son peuple. \textbf{La liberté d'expression ne signifie pas le droit de dénigrer ou de critiquer sans fondement}. La propagande basée sur le mensonge ou le dénigrement ne sera en aucun cas acceptée.

Le projet \ngelmak donnera à chacun les moyens de partager ou de dénoncer ce qu'il vit anormalement au quotidien, que ce soit au travail, dans son quartier ou chez lui. Pour que cette petite voix qui sommeille en nous puisse être modulée, transportée sur un bon canal de propagation, sûr et fiable, jusqu'aux récepteurs que sont, pour la plupart, les nombreux sénégalaises et sénégalais du monde entier. Cette petite voix pourra alors résonner comme une synergie dans les oreilles de tous les sénégalaises et sénégalais et de tous les acteurs internationaux.

Nous savons que ce projet s'annonce très laborrieuse et giganteste. Nous rencontrerons plusieurs obstacles et critiques venant de toutes part. Toutefois, nous ne flanchirons pas. Nous metrons toute notre énergie et savoir-faire pour l'implémentation de bout-en-bout et la réussite de ce projet.

L'objectif de ce document n'est pas d'entrer dans les détails techniques du projet, mais plutôt de donner un bref aperçu de ses différentes composantes. Dans les lignes qui suivent, nous aborderons les aspects techniques du projet d'un point de vue informatique, en décrivant les différentes étapes et les outils et technologies que nous utiliserons pour le mener à bien. Nous aborderons également les aspects juridiques et éthiques du projet en précisant les différents acteurs à impliquer dans sa mise en oeuvre. Enfin, nous nous pencherons sur les différentes perspectives que ce projet pourrait également couvrir dans les moments qui suivront sa publication.


% Le but de ce document n'est donc pas de revenir sur les différents scandales politiques, sociaux, économiques et religieux, mais de donner une contexte pour mieux cerner nos motivations. Nous proposons, en effet, totalement accessible et très efficace pour faire entendre la voix des personnes opprimées qui se sentent impuissantes face à l'injustice qu'elles-mêmes ou leurs proches subissent au quotidien. Cet outil permet également à tous ceux qui le souhaitent de:
% \begin{enumerate}
%   \item porter un message ouvert à son peuple ou à toute personnalité publique (prévention, dénonciation/critique, conseil/recommandation, etc,),
%   \item fournir une explication claire des actes, des décisions des autorités publiques ou de tout phénomène se déroulant dans l'espace public, et promouvoir la transparence vis-à-vis des décisions du gouvernement,
%   \item dénoncer:
%   \begin{itemize}
%     \item[-] toute violation des droits de l'homme au travail, dans le voisinage, dans le commerce et les échanges (par exemple, la fraude),
%     \item[-] la mauvaise gestion des services publics, l'abus de pouvoir, la corruption, etc.
%     \item[-] les troubles à l'ordre public: agressions, insécurité, vols, etc.
%   \end{itemize}
% \end{enumerate}


% Cependant, il est essentiel de savoir que le vrai reste au main du peuple. Toutefois, il est nécessaire que ce peuple en question sache sur quoi on le mene. Pour ce faire, les médias ne constituent pas toujours le meilleur moyen pour nous faire entendre. Sans mentionner ceux qui peuvent avoir la chance de passer sur leur chaine. Je  


% Le peuple doit avoir son canal d'information. Un canal par lequel passer en toute liberté, oui cette liberté même de la démocratie, pour faire passer son message et son désarois.


% Cette injuste accaparement du pourvoir au mains de la classe bourgeoise, fait naitre également d'autre injustice beaucoup plus silence que l'on se raconte en coulisse lors de nos longues assises de coserie.




% - Donner des avis sur le comportement de certaines entreprises pour lutter contre les faux avis.

% - Signaler une discrimination ou exaction

% - Proposer des activités sportives, éducatives (formation gratuite, conscientisation, etc),
