\chapter*{Preface}
\addcontentsline{toc}{chapter}{Preface} % Add the preface to the table of contents as a chapter

\textit{As Salamu Alaykum}, Bonjour à tous,\newline
Je m'appelle Youssouph FAYE et je suis ingénieur en informatique. Au moment où je commençais à écrire ces lignes pour la première fois, j'étais en deuxième année de thèse et je venais de fêter notre première vraie victoire en tant que peuple souverain du Sénégal avec le projet Pastef incarné par l'imminent Ousmane Sonko. Ce document a pour but de présenter le projet \projectname, qui signifie la cour ou la place publique en sérère.\newline
L'objectif de \projectname, que je vais détailler dans ce petit manuscrit, est d'offrir un outil accessible gratuitement à tous les sénégalaises et sénégalais, leur permettant d'exprimer leurs opinions et leurs critiques sur tous les aspects de la vie publique, dans le respect total des droits des uns et des autres. 
% En fait, l'idée d'un projet de volontariat m'habite depuis mon plus jeune âge. Je me suis toujours demandé comment je pouvais aider les autres avec le peu que j'possède. Cette idée est née d'une série de phénomènes et d'expériences que j'ai vécus pendant mon court séjour dans ce petit monde.

Alors, vous qui êtes en train de lire ces lignes, pour faire simple, l'idée est d'offrir une plateforme gratuite à 100\% pour permettre à chacun de s'exprimer et de signaler tout abus ou maltraitance. Vous, un membre de votre famille, votre voisin de palier, votre ami, pourriez être victime d'un de ces abus, alors vers qui vous tourneriez-vous pour en parler ? Vous diriez certainement la police. Peut-être, et c'est une excellente solution, que je valide sans hésitation. Mais si vous avez affaire à une holigarchie, à un système bien huilé d'oppression et d'exploitation des faibles, vous n'êtes pas au bout de vos peines. Et souvent, certains problèmes sont très isolés les uns des autres, ce qui minimise leur importance. Alors vous me direz, il y a beaucoup de gens vers qui se tourner comme Kewulo ou Guy Marius Sagna. Oui, bien sûr, ce sont de très grandes personnes qui aident les opprimés, les sans-voix et les impuissants. Cependant, il ne doit pas être facile pour eux de porter tous les gens sur leurs ailes et il n'est pas facile non plus de les rejoindre pour obtenir de l'aide [je suppose].

C'est précisément l'une des nombreuses raisons qui m'ont poussé, avec beaucoup d'autres personnes désireuses de m'aider, à me (nous) lancer dans ce projet, cette grande aventure dont les premiers jalons seront observables à la loupe, voire au microscope, mais qui suffiront amplement à porter la voix de centaines, de milliers, voire de millions de sénégalais à travers le monde. Mais en définitive, nous disposerons d'une solution plus complète, capable de couvrir un plus grand nombre de cas d'utilisation.

\hspace{1cm}

\begin{flushright}
	\textit{Youssouph FAYE}
\end{flushright}
