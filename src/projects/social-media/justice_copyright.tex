
\section{Questions Judiciaires}

Dans ce chapitre, j'aborderai les questions juridiques. Comme vous le savez peut-être, ce projet devra traiter une grande variété de données, ce qui nécessite de situer les responsabilités. Je dois cependant préciser que je ne suis pas un spécialiste de la justice. Il se peut donc que je commette des erreurs d'interprétation de certains textes ou que j'oublie tout simplement de mentionner des éléments fondamentaux. Aussi, n'hésitez pas à m'aider pour une couverture plus complète de ce point.


\subsection{\textsc{Question Des Données Personnelles}}

Au sénégal cette question est encadrée par la LOI n° 2008-12 du 25 janvier 2008 portant sur la Protection des données à caractère personnel.~\footnote{La Commission de Protection des Données Personnelles (CDP) s'assure que tout ce qui permet d'identifier une personne physique soit sécurisé et confidentiel. La loi portant sur la protection des données à caractère personnel est disponible à l'adresse~\href{https://www.cdp.sn/sites/default/files/protection.pdf}{LOI n° 2008-12 du 25 janvier 2008 portant sur la Protection des données à caractère personnel}}.

Données à caractère personnel: toute information relative à une personne physique identifiée ou identifiable directement ou indirectement, par référence à un numéro d'identification ou à un ou plusieurs éléments, propres à son identité physique physiologique, génétique, psychique, culturelle, sociale ou économique (LOI n° 2008-12 Article 4.6).

Dans le cadre de la mise en place d'une application au service du peuple sénégalais en premier lieu, nous veillerons toujours à ce que les données de nos utilisateurs ne soient pas utilisées à des fins personnelles ou commerciales. Ces données seront protégées et utilisées conformément à la loi régissant leur utilisation. L'utilisation des données personnelles des utilisateurs sera soumise à leur consentement et sera utilisée exclusivement pour répondre aux besoins primaires du projet. Nous nous engageons à respecter scrupuleusement la vie privée de nos utilisateurs. 


\subsection{\textsc{Modalité De Traitement De Données à Caractères Personnels}}

Certaines données personnelles telles que l'adresse électronique, la carte d'identité (pour certifier un compte) doivent être fournies, mais uniquement pour des raisons de commodité d'utilisation et de sécurité. Ces données personnelles ne sont utilisées que pour sécuriser les comptes (typiquement pour envoyer un lien de réinitialisation en cas d'oubli du mot de passe), ou pour contacter les utilisateurs en cas d'événement important. Nous ne les utiliserons pas à d'autres fins et ne les communiquerons pas à des tiers. Vous pourrez également supprimer votre compte facilement, à tout moment, depuis votre espace de paramétrage.

Tout contenu, tel qu'une publication ou un commentaire, appartient à son auteur et en est le seul propriétaire, de même que tous les éléments qui peuvent y être associés, tels que les vidéos, les sons, les textes, les fichiers, etc. Ils sont sous l'entière responsabilité de l'auteur, qui décide seul des modalités de leur mise à disposition du public, sous réserve de ne pas enfreindre nos politiques d'utilisation. La loi sur la protection des données privées encadre également les données produites et est soumise à une diffusion par son article 3.1. Ces publications, lorsqu'elles sont publiques (définies ci-dessous), feront l'objet d'un traitement automatisé pour de simples mesures d'organisation et la présentation de statistiques afin de valoriser les sujets traités. Dans certains traitements visant à établir des statistiques, par exemple, certaines données à caractère personnel, telles que le sexe, l'âge et la localisation, peuvent être utilisées avec le consentement de leur propriétaire, conformément à l'article 4.4 sur le consentement.

La durée de conservation des données personnelles d'un utilisateur et de tout contenu qu'il a créé est soumise à la volonté du propriétaire lui-même. \textbf{Tous les utilisateurs peuvent facilement supprimer certaines de leurs données, ou même leur compte, à tout moment, à partir des interfaces réservées à cet effet, sans en conserver une copie au-delà du délai de récupération.}.~\footnote{Le délai de récupération s'applique uniquement à la suppression d'un compte et donne à l'utilisateur un délai après avoir pris la décision de supprimer son compte pour le récupérer. Cette période ne dure pas plus d'un mois. Passé ce délai, le compte en question sera définitivement supprimé de l'application et aucune trace ne sera conservée.}




% 
% 
% 
% 
% A CHAPTER
% 
\section{Mentions Légales Et Conditions D'Utilisation}\label{copyright}

\subsection{\textsc{Mentions Légales}}

Nous sommes engagés à lutter contre l'oppression, à dénoncer l'abus de pouvoir et à promouvoir la justice. Notre projet à but non lucratif est dévouée à soutenir les personnes marginalisées et dominées et à responsabiliser les gouvernements en matière de gestion des ressources du peuple qu'ils représentent. Notre mission est ancrée dans l'eveil des consciences et dans l'action pour un monde plus juste et équitable pour tous.

\subsection{\textsc{Une Utilisation Respectant Les Normes Et Morales}}

Notre projet se veut de donner un moyen simple et efficace au peuple et lanceur d'alertes de s'exprimer, dénoncer et d'éveiller ses concitoyens. Cependant, cette utilisation devra respecter quelques normes pour son bon fonctionnement et le respecter de tous.

\begin{enumerate}
  \item Les utilisateurs s'engagent à utiliser l'application de manière responsable et éthique, en respectant les lois et réglements en vigueur au Sénégal.
  \item L'application ne doit pas être utiliser à des fins illégales, diffamatoires, trompeuse, obscènes, offensantes ou frauduleuses.
  \item Les utilisateurs consentent de ne pas publier des contenus faux, trompeurs ou manipulé intentionnellement pour nuire à autrui.
  \item Les utilisateurs s'engagent à respecter les droits d'auteur et les droits de propriété intellectuelle des tiers lors de la publication de contenu sur l'application.
  \item Tout contenu publié dans l'application doit impératiment respecter la vie privée et la dignité des individus, et ne doit en aucun cas violer leur droit.
  \item Les utilisateurs reconnaissent que la publication de preuves telles que des vidéos, des images ou des document doit être effectuée avec le consentement des parties concernées, sauf si cela est légalement autorisé dans le cadre de la dénonciation d'infraction.
  \item Les administrateurs et modérateurs se réservent le droit de supprimer tout contenu jugé contraire à ses politiques ou aux lois en vigueur, sans préavis.
  \item Les utilisateurs comprennent et acceptent que toute utilisation abusive peut entrainer des mesures de discipline, y compris la suspension ou la suppression de leur compte.
\end{enumerate}

Cependant, il existe également des mesures pour prévenir tout pouvoir répressif illégal envers les utilisateurs en cas de malentendu, de divergence d'option ou de croyances:

\begin{enumerate}
  \item Tout utilisateur dont le contenu est soumis à validation par des modérateurs volontaires doit avoir la possibilité de contester toute décision de modération jugée injuste ou arbitraire.
  \item Les procédures de validation des publications doivent être transparentes et équitables à l'aune des objectifs même du projet, avec la possibilité pour l'utilisateur de présenterdes arguments en sa faveur.
  \item En cas de contestation, un processus d'appel doit être mis en place, permettant à l'utilisateur de soumettre un ticket de réclamation visible au public et d'obtenir une réévaluation de sa publication.
  \item L'application doit fournir dans la mesure du possible des informations sur le(s) motif(s) de rejet afin que l'utilisateur puisse comprendre les raisons de la décision de médération.
  % \item En cas de litige concernant la modération du contenu, nous nous engageons à coopérer avec tous les différents acteurs (la société civile, organisations non gouvernementales, autorités judiciaires, etc.) et à fournir toutes les informations nécessaires pour résoudre le différent de manière légale et équitable.
  \item Les modérateurs volontaires doivent être suivis et contrôlés pour garantir qu'ils agissent de manière impartiale et respectent les droits utilisateurs, en évitant toute discrimination basée sur les opinions ou les croyances.
  \item En cas de litige persistant concernant la modération d'un contenu, l'utilisateur doit avoir la possibilité d'engager des procédures judiciaires par un arbitrage pour faire valoir ses droits, ceci afin d'éviter de s'attributer la vérité absolue.
  \item Nous nous engageons à respecter les décisions provenant d'autorités judiciaires ou d'arbitre concernant la modération du contenu et à prendre des mesures nécessaires pour se conformer à celles-ci.
\end{enumerate}

Le non respect de ses normes et régelements peut faire l'object à des sanctions.

Certains autres lois peuvent également être concernées dans la porté de ce projet. De ce fait, nous nous engageons et prendrons toutes les mesures qu'il faut pour respecter et faire respecter toutes les lois qui regissent notre pays.


\subsection{\textsc{Droit de propriété (GNU GPL copyleft)}}


Le projet \ngelmak, ou l'application qui en est dérivée, sera libre et gratuit. Toutes ses versions seront publiques. Ce projet est soumis à la version 3 de la licence publique générale GNU ou GNU General Public License (GPLv3). Cette licence garantit que le logiciel sera un logiciel libre et le restera, quelle que soit la personne qui modifie ou distribue le programme. Vous trouverez plus de détails sur cette licence sur sa page~\href{https://www.gnu.org/licenses/quick-guide-gplv3.html}{https://www.gnu.org/licenses/quick-guide-gplv3.html}.