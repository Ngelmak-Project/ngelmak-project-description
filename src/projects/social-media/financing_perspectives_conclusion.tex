\section{Source De Financement Du Projet}\label{chap:finance}

Nous mettrons en place un certain nombre de programmes de financement basés sur la participation volontaire et le soutien de diverses parties prenantes afin que notre application reste accessible.


% SECTION
\subsection{Modèle de Financement: Qui Peut Nous Soutenir ?}

\paragraph{Une participation volontaire.}

J'invite tous les acteurs à participer en commençant par moi-même en tant que premier donateur qui est prêt à donner son temps et ses ressources telles que l'argent et les compétences pour la réussite de ce projet. Il est important de savoir que chaque don, aussi petit soit-il, peut aider à couvrir les coûts de développement, de maintenance et d'amélioration de l'application. Chaque don, qu'il soit en nature ou en espèces, compte, et nous invitons tous les utilisateurs à s'impliquer et à nous aider à faire de ce projet une réalité.

Nous avons besoin de tout type de muscle pour nous aider. Si vous êtes développeur ou informaticien comme moi et que vous pensez que vous pourriez être utile et que vous voulez montrer ce dont vous êtes capable pour aider les gens, nous avons besoin de vous. Si vous êtes juriste et que vous pensez pouvoir apporter une aide quelconque, en particulier sur les questions juridictionnelles, nous avons besoin de vous. Si vous êtes dans une autre spécialité et que vous pensez pouvoir être utile d'une manière ou d'une autre, par exemple en étant candidat comme volontaire pour la modération ou en aidant les utilisateurs à utiliser l'application, alors nous avons définitivement besoin de vous.

En rassemblant tous nos efforts, nous pouvons créer quelque chose de puissant qui aidera ce projet à atteindre son objectif. Nous ne pourrions pas l'atteindre sans le soutien mutuel de tous.

\paragraph{Le Soutien De Nos Institutions.}

Nous chercherons à obtenir des subventions et un soutien financier de la part des pouvoirs publics. Par la présentation de notre projet comme un outil essentiel de soutien aux opprimés et de promotion de la justice sociale, nous espérons obtenir un soutien significatif de leur part.

En même temps, nous espérons bénéficier de partenariats avec toutes les organisations non gouvernementales et autres groupes qui partagent nos valeurs et adhèrent à nos objectifs. Nous espérons cependant établir des partenariats constructifs, que ce soit sous la forme d'un soutien financier, de conseils ou de fourniture de ressources et de services techniques.

Cependant, aucune subvention destinée à soutenir une quelconque affaire politique ne sera acceptée. De plus, tout financement provenant d'une organisation qui ne respecte pas nos valeurs éthiques et nos politiques sera rejeté. Nous ne cautionnerons aucune injustice et ne trahirons pas les valeurs qui guident les personnes que nous assistons par ce présent projet.

\paragraph{Le secteur privé.}

Les entreprises et les acteurs du secteur privé peuvent également apporter leur aide par le biais du sponsoring. Le sponsoring leur permet de démontrer leur engagement en faveur de causes sociales et des droits de l'homme.

Nous solliciterons des dons et des contributions auprès de notables, de philanthropes et de responsables communautaires. Leur soutien sera inestimable pour la visibilité et la durabilité du projet.


\subsection{Transparence : Origine Et Destination Des Fonds}

Nous nous engageons à une transparence totale quant à l'utilisation des fonds qui seront collectés pour ce projet par le biais de dons. À cette fin, nous mettrons en oeuvre :

\paragraph{Rapports financiers accessibles au public.}

Des rapports financiers détaillant les dépenses et les sources de financement seront publiés régulièrement sur notre plateforme. Les utilisateurs pourront suivre l'utilisation des fonds ainsi que leur origine, ce qui leur permettra de savoir qui donne quoi et quand. Cela démontrera notre bonne foi envers nos utilisateurs et donateurs et l'impact de leurs dons sur l'avancement du projet.

\paragraph{Comité de surveillance.}

Nous restons ouverts à tout comité indépendant souhaitant superviser la gestion financière et s'assurer que les fonds sont utilisés de manière éthique et efficace.


% 
% 
% 
% 
% 
% CHAPTER
% 
\section{Perspectives et Conclusion}

\subsection{Les Perspectives à Venir}

\paragraph{Le covoiturage comme solution de transport.}

Plusieurs systèmes d'autopartage existent dans le monde. En France, par exemple, il y a \textit{BlaBlaCar} même s'il s'agit d'une société commerciale. Nous pourrions envisager cette fonction pour permettre à tout conducteur de partager sa voiture avec d'autres personnes, soit gratuitement, soit moyennant une rémunération (cotisation) convenue d'un commun accord. Des usagers voyageant occasionnellement sur de longues distances peuvent également proposer des itinéraires et entrer en contact avec d'autres voyageurs désireux de partager le même trajet.

Tout cela se fera sans aucune forme de tarification de notre part. En fait, cette fonctionnalité sera gratuite, nous ne facturerons rien pour tirer parti de cette fonctionnalité. Il appartient aux utilisateurs ou aux bénéficiaires de faire généreusement des dons à la plateforme si ce travail les a aidés d'une manière ou d'une autre à soutenir ce travail. Pour des raisons de sécurité, tous les voyageurs doivent toutefois faire vérifier leur identité par l'application avant de profiter de cette fonctionnalité.

\paragraph{Autres.}

Des fonctionnalités supplémentaires pourraient être ajoutées au fur et à mesure du développement. Une bonne idée serait d'intégrer le commerce d'objets d'occasion (bon plan) entre les utilisateurs et les activités des associations, le tout gratuitement. Nous pourrions également créer une association qui collecte des dons de livres et les prête gratuitement à d'autres lecteurs.

Ce projet est basé sur l'utilisation gratuite de tous ses services, et nous laissons les gens libres de donner ce qu'ils peuvent, soit aux personnes derrière le travail, soit à l'application (ou au projet) lui-même. Pour nous, le simple plaisir de voir son travail profiter à d'autres est un grand réconfort. Nous développerons ce projet nous-mêmes, nous le financerons à ses débuts sur nos propres deniers, comme notre propre enfant, aussi longtemps que nous le pourrons. Le projet \ngelmak ne survivra certainement pas sans la participation de tous et l'effort collectif.

\subsection{Conclusion}

Nous voici arrivés au terme de ce manuscrit qui détaille vaguement le projet \ngelmak. Ce document a plus ou moins essayé de décrire les principes de base et les fonctionnalités de notre projet. Nous nous mettrons au travail dès la publication de ce document, ou attendrons un répit pour recueillir des commentaires et constituer une équipe de volontaires (informaticiens, juristes, etc.). \textbf{Chaque version stable de \ngelmak sera publique et open-source}. Le projet \ngelmak participe au scillage pour une science ouverte. Toute copie de ce projet, en tout ou en partie, doit rester accessible à tous, sans aucune forme de monétisation.

Nous aimerions conclure en faisant appel à tous ceux qui sont prêts à nous aider ou à nous accompagner dans la mise en place de ce projet. Nous nous adressons à toutes les personnes intéressées, telles que:
\begin{itemize}
  \item[-] les \textbf{informaticiens} qui veulent aider à la mise en place du projet grâce à leurs compétences,
  \item[-] les \textbf{personnalités juridiques} qui souhaitent à la fois nous aider à traiter les questions juridiques que ce projet peut soulever et à mieux définir nos conditions d'utilisation,
  \item[-] toute personne souhaitant se porter \textbf{volontaire} en tant que membre de l'équipe chargée de la modération et de l'assistance aux utilisateurs,
  \item[-] tout \textbf{donateur} qui souhaite nous soutenir financièrement ou matériellement,
  \item[-] toute personne qui peut \textbf{faire connaître} ce projet à d'autres personnes via leur réseau personnel ou professionnel (WhatsApp, Facebook, Instagram, etc.) afin qu'il atteigne le plus grand nombre de personnes possible,
  \item[-] etc.
\end{itemize}

Vous voyez, peu importe qui vous êtes et peu importe votre rôle dans la société, vous pouvez participer à ce projet sénégalais par des sénégalais et pour des sénégalais [et, à terme, pour l'ensemble de l'Afrique].


\paragraph{Contacts}

\begin{table}[ht]
  \centering
  \resizebox{\textwidth}{!}{%
  \begin{tabular}{ll}
  \textbf{Email}            & ngelmak@proton.me \\
  \textbf{Telephone}        & +221 77 584 45 24 (whatsapp) \\
  \textbf{Groupe WhatsApp}  & https://chat.whatsapp.com/Eytx81Vb4CpGBhr8ntrdnW         
  \end{tabular}%
  }
\end{table}



\begin{center}
  \Large{
    \textbf{
      \texttt{
        Toute critique ou suggestion d'amélioration est la bienvenue.
      }
    }
  }
\end{center}