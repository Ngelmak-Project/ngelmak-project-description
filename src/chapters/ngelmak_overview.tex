% ** A NEW CHAPTER **
\NgelmakProjectNewChapter{Vision Générale, Motivation et Fondements du Projet \ngelmak}

\section{Contexte géopolitique et enjeux de la communication africaine}

La communication occupe aujourd'hui une place centrale dans les dynamiques géopolitiques mondiales. Pourtant, nous, Africains, continuons de dépendre de canaux d'information conçus et contrôlés par des acteurs extérieurs, qui imposent leur propre narratif sur notre continent. Cette dépendance structurelle leur permet de façonner la perception des événements qui se déroulent chez nous, au point que nous nous retrouvons souvent noyés dans la désinformation, la manipulation et la propagande anti-africaine. Ce phénomène est devenu si courant qu'il ne nécessite même plus d'exemples : il constitue désormais une réalité quotidienne pour des millions d'Africains qui, parfois, connaissent à peine leurs propres terres et leurs propres histoires.

À cette domination narrative s'ajoute une forme de censure systématique. Dès que nous proposons un récit différent de celui imposé par des acteurs hostiles à nos intérêts, nos voix sont filtrées, marginalisées ou réduites au silence. Ce déséquilibre est aggravé par l'absence d'alternatives crédibles proposées par nos dirigeants, nos institutions régionales ou continentales. Ni les États, ni les organisations sous-régionales, ni même l'Union africaine ne mettent en place des solutions souveraines permettant aux Africains de communiquer entre eux, avec leurs peuples ou avec le reste du monde. Pourtant, de nombreuses technologies open source existent déjà et pourraient être adaptées à nos besoins.

Malgré cela, nous continuons à utiliser des plateformes étrangères pour exprimer nos opinions, partager nos idées ou débattre de nos enjeux. Cette dépendance nous fragilise, car elle place notre parole, notre mémoire collective et notre capacité d'analyse entre les mains d'acteurs dont les intérêts ne coincident pas nécessairement avec les nôtres. Dans un monde où l'information est devenue un instrument de pouvoir, cette situation constitue un handicap majeur pour notre souveraineté.

\section{Nécessité d'un réseau souverain et motivations du projet \ngelmak}

Face à ce constat, la nécessité de construire un réseau conçu par des Africains et destiné aux Africains apparaît avec évidence. C'est dans cet esprit qu'est né le Projet \ngelmak. \ngelmak, qui signifie \textit{cour} en sérère, renvoie à cet espace traditionnel de rassemblement où l'on échange, débat et construit ensemble, et c'est précisément ce rôle que nous voulons redonner à notre espace numérique souverain. Nous devons disposer d'un environnement capable de transmettre notre narratif, de documenter notre présent et de préserver notre mémoire collective. Un espace où nous pouvons partager nos idées, analyser nos réalités et débattre de nos enjeux sans craindre la censure, la condescendance ou la manipulation.

Ce projet vise à nous sortir de la posture réactive dans laquelle nous sommes enfermés depuis trop longtemps. Nous passons notre temps à dénoncer le racisme, la manipulation médiatique, l'ingérence ou la désinformation, sans jamais disposer d'un outil qui nous permette de proposer notre propre vision du monde. Les technologies existent déjà ; il nous appartient désormais de les utiliser pour créer, à travers \ngelmak, un environnement numérique qui serve nos intérêts et reflète nos valeurs.

Nous sommes conscients que le démarrage sera modeste et vulnérable. Nous ne rivaliserons pas immédiatement avec les grandes plateformes mondiales. Nous affronterons des difficultés, nous connaîtrons des échecs, mais nous apprendrons, nous progresserons et nous nous renforcerons. Ce processus est nécessaire pour bâtir un outil durable, crédible et utile à long terme. Notre détermination doit être totale, car ce projet s'inscrit dans une démarche de souveraineté, de dignité et de responsabilité collective.

Cette initiative dépasse nos ressources individuelles, mais elle s'inscrit dans une volonté commune de contribuer à la renaissance africaine. Nous devons cesser de chercher la reconnaissance dans des institutions qui ne nous servent pas. Nous devons nous imposer par des réalisations concrètes, par des faits, par des outils comme \ngelmak, qui démontrent notre capacité à construire, à innover et à nous organiser. L'Afrique dispose de tout le potentiel nécessaire pour avancer ; il nous revient de l'activer.

\section{Vision panafricaine, unité et principes éthiques}

\ngelmak s'inscrit dans une vision panafricaine fondée sur l'unité, la dignité et la souveraineté intellectuelle. Nous sommes conscients que la majorité des pays africains ont été artificiellement créés par des puissances étrangères, sans lien avec nos peuples ni nos réalités. Cette fragmentation imposée a laissé des traces profondes, tant physiques que spirituelles. Nous refusons que cette division historique continue de nous séparer dans notre manière de penser, de communiquer ou de nous percevoir collectivement.

Malgré nos différences idéologiques, linguistiques, religieuses, culturelles ou sociales, nous partageons une condition commune : nous sommes souvent perçus de l'extérieur comme un ensemble homogène, réduit à des stéréotypes déshumanisants. Ce projet n'a pas pour objectif de convaincre qui que ce soit du contraire, mais de nous permettre de nous rassembler, de construire notre propre espace et de produire notre propre récit. Il s'agit de reprendre la maîtrise de notre parole, de notre image et de notre histoire.

Dans cette perspective, nous établissons des principes éthiques clairs. Aucune forme de condescendance, d'insulte ou de dénigrement ne sera tolérée. Les expressions visant à réduire au silence, à diviser ou à humilier — telles que "occupez-vous de votre pays" ou toute variante similaire — n'auront pas leur place dans cet espace. Nous voulons créer un environnement où chacun peut s'exprimer avec respect, rigueur et responsabilité, dans l'intérêt collectif.

Nous affirmons également que notre projet doit permettre aux Africains de parler de l'Afrique, mais aussi du monde entier. Les dynamiques globales nous concernent, car elles influencent directement nos sociétés, nos économies et nos trajectoires. Nous ne devons pas nous limiter à commenter ce qui se passe chez nous : nous devons participer pleinement aux discussions internationales, avec nos propres analyses et nos propres perspectives.

Cette vision éthique et panafricaine constitue le socle sur lequel repose l'ensemble du projet \ngelmak. Elle guide nos choix, nos priorités et notre manière de concevoir les outils que nous mettons en place. Elle nous rappelle que notre objectif n'est pas seulement technique ou fonctionnel, mais profondément humain, culturel et politique au sens noble du terme : contribuer à l'éveil, à la cohésion et à la souveraineté de nos peuples.

\section{Autonomie et financement du projet \ngelmak}

Le financement du projet \ngelmak repose sur une approche simple, transparente et entièrement indépendante. Le développement initial est assuré par les ressources personnelles du fondateur et des membres les plus investis. Ce choix traduit la volonté de préserver l'autonomie du projet et d'éviter toute influence extérieure susceptible d'en altérer les valeurs ou la vision.

À mesure que la plateforme évoluera, des dons volontaires pourront être sollicités auprès de la communauté. Ces contributions resteront toujours facultatives et ne conditionneront jamais l'accès aux services. Elles auront pour seule finalité de couvrir les coûts techniques, matériels et humains nécessaires au maintien et à l'amélioration de l'outil. Aucun service payant, aucune fonctionnalité premium et aucune forme de commission ne seront proposés. Le projet n'a pas vocation à générer un profit : il repose sur un modèle de solidarité et de participation collective.

Cette approche garantit que l'ensemble des services restera accessible à tous, sans distinction de moyens, et que le projet demeurera libre de toute pression économique, politique ou commerciale. L'objectif n'est pas de créer une entreprise, mais de bâtir un outil communautaire, durable et au service de l'intérêt général.



\section{Fondements sacrés du projet \ngelmak}

\ngelmak repose sur un ensemble de principes intangibles, considérés comme sacrés et non négociables. Ces fondements définissent son identité, orientent sa gouvernance et garantissent sa fidélité à la vision qui l'a fait naître. Ils constituent la base éthique et politique sur laquelle repose l'ensemble de l'initiative.

\paragraph{Gratuité absolue.}
Tous les services proposés par le projet resteront toujours gratuits, sans aucune exception. La formule souvent répétée — "si c'est gratuit, c'est vous le produit" — ne s'applique pas ici. Nous rejetons toute forme de monétisation directe ou indirecte des utilisateurs : aucune fonctionnalité ne sera payante, aucune commission ne sera prélevée, et aucun accès ne sera conditionné à un abonnement ou à un paiement. La gratuité n'est pas un argument commercial, mais un engagement éthique fondamental, garantissant que l'outil demeure un bien commun accessible à tous.

\paragraph{Absence totale de volonté de profit.}
\ngelmak n'a pas vocation à générer des revenus, à attirer des investisseurs ou à se transformer en entreprise lucrative. Il s'agit d'un outil communautaire conçu pour répondre à des besoins collectifs, et non pour servir des intérêts privés. Cette orientation exclut toute forme de publicité intrusive, de vente de données, de partenariats commerciaux contraires à nos valeurs ou de modèle économique visant à tirer profit des utilisateurs. \ngelmak ne poursuivra jamais un objectif financier : il est et restera non lucratif.

\paragraph{Protection stricte des données.}
Aucune donnée ne sera vendue, partagée ou exploitée par des tiers, sous quelque forme que ce soit. Les informations confiées par les utilisateurs ne seront ni échangées, ni utilisées à des fins commerciales, politiques ou publicitaires. La confidentialité et la sécurité des données constituent des engagements absolus, respectés à chaque étape du développement, du fonctionnement et de l'évolution du projet. La protection des utilisateurs n'est pas une fonctionnalité : c'est un principe fondateur.

\paragraph{Gouvernance centrée sur la communauté.}
Toute évolution majeure des politiques, des règles fondamentales ou des orientations générales devra être soumise à la communauté. \ngelmak n'appartient pas seulement à ses fondateurs : il appartient à celles et ceux qui l'utilisent et le font vivre. Cette gouvernance participative garantit que les décisions importantes ne pourront jamais être prises de manière unilatérale ou contre l'intérêt collectif. Elle assure la continuité du projet, sa stabilité et sa fidélité à ses valeurs initiales.

\paragraph{Transparence totale.}
Les décisions importantes, les évolutions techniques, les orientations stratégiques et les choix de gouvernance seront documentés et accessibles à tous. Cette transparence s'applique également à l'usage des dons : chaque contribution financière fera l'objet d'un suivi clair, public et vérifiable, permettant à la communauté de connaître précisément l'affectation des ressources. Rien de ce qui concerne le fonctionnement, le financement ou l'avenir du projet ne sera dissimulé ou décidé dans l'ombre. La transparence constitue un pilier essentiel de la confiance, de la légitimité et de la pérennité du projet.


Ces fondements sacrés définissent ce que nous sommes, ce que nous voulons construire et ce que nous refusons de devenir. Ils garantissent que \ngelmak restera un espace libre, éthique, souverain et profondément ancré dans l'intérêt général.


\section{Droits, propriété et gouvernance du projet}

\ngelmak repose sur un cadre juridique et organisationnel destiné à garantir son ouverture, sa pérennité et la protection de sa vision fondatrice. L'ensemble du code source, de la documentation, de l'architecture et des éléments conceptuels a été initié, conçu et développé par le fondateur du projet, qui en détient les droits d'auteur initiaux. Ces droits constituent la base légale du projet et définissent son identité.

Le projet \ngelmak est publié sous une licence open source de type GNU, garantissant que le code restera libre, accessible et modifiable par tous. Cette licence protège \ngelmak contre toute tentative de privatisation, d'appropriation ou de détournement. Elle assure également que toute contribution future restera compatible avec les principes de liberté, de transparence et de partage qui fondent le projet.

La gouvernance du projet repose sur trois niveaux complémentaires.

\paragraph{Fondateur.}
Le fondateur est le détenteur du copyright initial et le garant de la vision. Il conserve l'autorité finale sur les orientations stratégiques, la validation des versions officielles, la définition des règles fondamentales et la protection de l'identité du projet. Ce rôle est permanent et ne peut être retiré, transféré ou remis en cause par aucune instance interne ou externe. Le fondateur demeure le gardien ultime de la cohérence, de la continuité et de la fidélité du projet à ses fondements sacrés.

\paragraph{Contributeurs.}
Les contributeurs regroupent toutes les personnes participant au développement, à la rédaction, à la conception ou au financement. Leur participation est reconnue, valorisée et intégrée dans le projet. Ils peuvent proposer des améliorations, développer des modules, rédiger des documents ou contribuer à la maintenance. Toutefois, leurs contributions s'inscrivent dans le cadre défini par les principes fondateurs, et toute modification majeure nécessite la validation du fondateur.

\paragraph{Communauté.}
La communauté occupe une place centrale dans la gouvernance. Elle peut proposer des idées, voter sur certaines orientations, signaler des problèmes, participer à la réflexion collective et contribuer à l'évolution du projet. Aucun changement de politique, aucune modification des fondements sacrés et aucune évolution majeure ne pourra être adoptée sans consultation de la communauté. Toutefois, cette participation ne remet pas en cause le rôle du fondateur, qui demeure l'autorité ultime chargée de préserver la cohérence et l'intégrité du projet.

Ce cadre de propriété et de gouvernance assure que \ngelmak restera un bien commun, libre, éthique et durable, tout en protégeant la vision initiale contre toute tentative d'appropriation ou de déviation.


\section{Clarification des décisions stratégiques}

Dans le cadre de ce projet, une décision stratégique désigne toute décision susceptible d'affecter la vision, l'identité ou les fondements du projet. Cela inclut notamment :

\begin{itemize}
    \item la modification des principes sacrés (gratuité, absence de profit, protection des données) ;
    \item l'introduction de publicité, de commissions ou de services payants ;
    \item tout changement de licence ou de modèle économique ;
    \item toute tentative de privatisation, de commercialisation ou de mise en bourse ;
    \item toute modification de la gouvernance ou des droits fondamentaux ;
    \item toute évolution susceptible de détourner \ngelmak de ses objectifs initiaux.
\end{itemize}

Ces décisions nécessitent obligatoirement la consultation de la communauté, la validation du fondateur et le respect strict des fondements sacrés. Le rôle du fondateur n'est pas un pouvoir absolu, mais un pouvoir de protection destiné à garantir que \ngelmak ne pourra jamais être détourné, privatisé ou transformé en produit commercial. Les décisions techniques, opérationnelles ou fonctionnelles restent ouvertes à la participation collective.


% \section{Perspectives d'évolution de la gouvernance}

% Pour renforcer la structure du projet et garantir sa pérennité, plusieurs éléments pourront être développés dans les phases ultérieures :

% \begin{itemize}
%     \item une charte de gouvernance complète définissant les rôles et responsabilités ;
%     \item une charte éthique détaillant les valeurs et les règles de comportement ;
%     \item un document de contribution (CONTRIBUTING) décrivant les processus de participation ;
%     \item un statut de projet communautaire formalisant les droits et devoirs de chacun ;
%     \item une licence GNU adaptée, éventuellement complétée par des clauses éthiques supplémentaires.
% \end{itemize}

% Ces outils permettront d'assurer la continuité, la transparence et la cohérence du projet tout en renforçant la participation de la communauté.


\section{Conclusion}

Ce chapitre a posé les bases conceptuelles, éthiques et organisationnelles du projet {\ngelmak{}}. Il a présenté les motivations profondes qui justifient sa création, les enjeux géopolitiques qui rendent indispensable l'émergence d'un espace numérique souverain, ainsi que les principes sacrés qui guideront son développement. \ngelmak s'inscrit dans une démarche de responsabilité collective, de transparence et de souveraineté, en affirmant clairement son refus de toute forme de privatisation, de commercialisation ou de compromission avec des intérêts extérieurs.

La structure de gouvernance définie ici garantit un équilibre entre l'ouverture à la participation communautaire et la protection de la vision fondatrice. Le rôle du fondateur, les responsabilités des contributeurs et la place centrale de la communauté ont été établis de manière à assurer la continuité, la cohérence et la fidélité du projet à ses valeurs. Les mécanismes de décision, en particulier ceux concernant les orientations stratégiques, ont été conçus pour empêcher toute dérive susceptible de détourner \ngelmak de ses objectifs initiaux.

Les fondements sacrés — gratuité absolue, absence de profit, protection stricte des données, transparence et gouvernance communautaire — constituent le socle immuable sur lequel reposera l'ensemble des outils et services développés. Ils garantissent que \ngelmak demeurera un bien commun numérique, accessible à tous et protégé contre toute tentative de déviation ou d'appropriation.

Ce cadre général ouvre la voie aux développements techniques, organisationnels et communautaires qui seront détaillés dans les chapitres suivants. Il marque le point de départ d'une initiative ambitieuse, construite avec rigueur et conviction, et destinée à servir durablement les intérêts des Africains et de tous ceux qui partagent cette vision.