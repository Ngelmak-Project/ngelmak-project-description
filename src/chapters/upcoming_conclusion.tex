\NgelmakProjectNewChapter{Les Prochaines Etapes du Projet \ngelmak}[images/upcoming/nadine-marfurt-kPUQmpH1U-E-unsplash.jpg][images/upcoming/polina-koroleva-YUUSeuO2Hmk-unsplash.jpg]


\section{Perspectives et développements à venir}

\paragraph{Consolider les fondations du Projet \ngelmak.}

Les premiers mois du Projet \ngelmak seront consacrés à la stabilisation des outils essentiels, à l'amélioration de l'expérience utilisateur et à la mise en place d'un environnement technique robuste. Il s'agira d'assurer la fiabilité des services existants, de renforcer la sécurité des données et de garantir une accessibilité optimale, même dans des zones où la connectivité reste limitée. Cette phase de consolidation est indispensable pour bâtir une base solide sur laquelle les futures fonctionnalités pourront s'appuyer.

\paragraph{Étendre progressivement les services.}

Une fois les fondations stabilisées, \ngelmak intégrera de nouveaux services destinés à répondre aux besoins concrets des communautés africaines. L'objectif n'est pas de multiplier les fonctionnalités pour impressionner, mais de proposer des outils réellement utiles, cohérents avec notre vision et adaptés aux réalités du terrain. Les priorités incluront l'amélioration du système de mobilité, l'intégration de services communautaires, la mise en place d'espaces de partage de connaissances et la création de mécanismes permettant aux utilisateurs de contribuer directement à l'évolution de la plateforme.

\paragraph{Renforcer la dimension communautaire.}

\ngelmak n'est pas seulement un outil numérique ; c'est un espace social, un lieu de rassemblement moderne inspiré de la \textit{cour} traditionnelle. Les prochaines étapes viseront à renforcer cette dimension communautaire en facilitant l'organisation de groupes locaux, en encourageant la participation citoyenne et en développant des outils de modération collective. L'objectif est de faire de \ngelmak un espace où chacun peut contribuer, apprendre, transmettre et s'engager dans la construction d'un environnement numérique souverain.

\paragraph{Développer une gouvernance ouverte et transparente.}

À mesure que le projet grandira, il deviendra nécessaire d'établir une gouvernance plus structurée, fondée sur la transparence, la participation et la responsabilité. Les utilisateurs devront pouvoir comprendre comment les décisions sont prises, comment les données sont protégées et comment les orientations futures sont définies. Cette gouvernance ouverte sera un pilier essentiel pour maintenir la confiance, éviter les dérives et garantir que \ngelmak reste un outil au service du bien commun.

\paragraph{Préparer l'interconnexion avec d'autres initiatives africaines.}

Le Projet \ngelmak ne se construit pas en isolation. À long terme, il devra s'interconnecter avec d'autres initiatives africaines, qu'elles soient technologiques, éducatives, culturelles ou économiques. L'objectif est de contribuer à un écosystème numérique africain cohérent, capable de rivaliser avec les grandes plateformes mondiales tout en préservant nos valeurs, nos langues et nos réalités. Cette interconnexion permettra de renforcer la souveraineté numérique du continent et d'ouvrir la voie à de nouvelles formes de collaboration.


\section{Conclusion Générale}

Le Projet \ngelmak s'inscrit dans une démarche de souveraineté, de dignité et de responsabilité collective. Il ne s'agit pas simplement de créer une plateforme numérique, mais de bâtir un espace qui nous appartient, qui reflète nos valeurs et qui répond à nos besoins. Dans un monde où les récits, les technologies et les infrastructures sont souvent contrôlés par d'autres, il devient essentiel de reprendre l'initiative et de construire nos propres outils.

\ngelmak, en tant que \textit{cour} moderne, se veut un lieu de rassemblement, de réflexion et d'action. Un espace où les Africains peuvent se retrouver, partager, apprendre et s'organiser sans dépendre de structures extérieures. Un espace où la mobilité, l'information, la culture et la communauté se rejoignent pour former un réseau souverain, accessible et durable.

Ce document n'est qu'une première étape. Le chemin sera long, parfois difficile, mais nécessaire. Nous avancerons avec humilité, détermination et lucidité. Nous ferons des erreurs, nous corrigerons, nous progresserons. Ce qui compte, c'est la direction : construire un avenir où l'Afrique ne se contente plus de réagir, mais propose, innove et s'impose par ses propres réalisations.

\ngelmak est une invitation à participer à cette construction. Une invitation à reprendre notre place dans le monde numérique. Une invitation à bâtir, ensemble, un espace qui nous ressemble et qui nous rassemble.     