% ** A NEW CHAPTER **
\NgelmakProjectNewChapter{Rendre les Déplacements plus Simples pour Tous}[images/transport/jacques-dillies-ITM82DCj6i8-unsplash.jpg][images/transport/joshua-oluwagbemiga-t6nqZ0n3i-k-unsplash.jpg]


\section{Constat et motivations}

La mobilité quotidienne reste l'un des défis les plus pénibles pour une grande partie des populations africaines. Pour aller au travail, rejoindre l'école, assister à un événement ou simplement rendre visite à un proche, il faut souvent traverser un système désorganisé, saturé d'intermédiaires, de pratiques abusives et d'attentes interminables. Le covoiturage, pourtant profondément ancré dans les habitudes depuis des décennies, n'a jamais bénéficié d'une structuration moderne ou d'un accompagnement numérique adapté aux réalités locales. Il fonctionne, mais dans un chaos permanent qui pénalise à la fois les voyageurs et les conducteurs.

Dans de nombreuses villes, la scène est toujours la même : pour trouver un trajet, il faut se rendre physiquement à un point de rassemblement connu de tous. Là, une multitude d'intermédiaires se disputent les voyageurs, les interpellent, les tirent par le bras, crient des destinations et imposent des prix arbitraires. Le conducteur, qui fournit réellement le service, voit sa marge réduite par ces acteurs parasites. Le passager, lui, paie plus cher, attend plus longtemps et subit un stress inutile. Ce système, hérité d'une longue histoire de débrouillardise et d'absence d'organisation publique, est devenu une norme que tout le monde subit sans jamais la remettre en question.

Les tentatives de numérisation existantes, lorsqu'elles apparaissent, se contentent souvent de remplacer les intermédiaires physiques par des plateformes privées qui imposent des commissions exorbitantes. Le résultat est le même : le voyageur paie trop, le conducteur gagne trop peu, et la plateforme s'enrichit. Rien n'est pensé pour les réalités africaines, pour les revenus locaux, pour les contraintes quotidiennes, ni pour la souveraineté numérique. Le problème n'est pas seulement technique : il est structurel, culturel et économique.

\section{Une alternative souveraine et non lucrative}

Face à ce constat, \ngelmak propose une alternative radicalement différente. L'objectif n'est pas de créer une énième plateforme de covoiturage cherchant à maximiser ses profits, mais de construire un service public numérique, souverain, gratuit et orienté vers l'intérêt collectif. Nous refusons la logique entrepreneuriale classique centrée sur le triptyque problème-profit-pour-soi. Notre démarche repose sur une philosophie simple : problème-aide-solution. Nous voulons démontrer qu'un groupe de citoyens peut construire un outil que les gouvernements auraient dû proposer depuis longtemps, un outil qui facilite la vie des gens au lieu de la compliquer, un outil qui aide les jeunes à générer un revenu complémentaire sans être exploités, un outil qui réduit les coûts de transport pour les usagers et fluidifie les déplacements.

Cette vision repose sur un principe fondamental : aucune commission, aucune tarification imposée, aucune exploitation des conducteurs ou des voyageurs. Le service est conçu pour être un bien commun numérique, accessible à tous, et pensé pour répondre aux réalités africaines plutôt qu'aux logiques de profit.

\section{Les usages principaux du service}

\subsection{Le conducteur qui propose un trajet}

Le premier usage du service concerne les conducteurs qui souhaitent proposer un trajet. Un particulier possédant un véhicule ou tout autre moyen de locomotion peut publier un trajet régulier, occasionnel, saisonnier ou spontané. Il peut fixer un prix, laisser la possibilité de négocier ou proposer gratuitement. Les voyageurs intéressés envoient une demande, et le conducteur reste libre d'accepter ou de refuser. Les points de départ et d'arrivée peuvent être fixes ou flexibles, permettant d'ajuster le parcours selon les besoins des passagers.

Ce modèle permet au conducteur de réduire ses coûts de carburant et d'entretien, d'optimiser ses déplacements, de partager la conduite lorsque cela est possible et de générer un revenu complémentaire sans subir la pression d'intermédiaires ou de commissions imposées. Il transforme un trajet déjà prévu en une opportunité d'entraide et de réduction des dépenses.

\subsection{Le voyageur qui initie une demande}

Le second usage concerne les voyageurs qui souhaitent initier une demande de trajet. Un utilisateur peut publier une requête en précisant l'heure, la date, le lieu de départ, le lieu d'arrivée et éventuellement un prix indicatif. Les conducteurs situés dans la zone concernée voient la demande et peuvent l'accepter, proposer un ajustement ou simplement passer leur tour sans aucune pénalité. Le voyageur choisit ensuite l'offre qui lui convient, et les autres propositions disparaissent automatiquement.

Ce modèle permet aux voyageurs de trouver un trajet rapidement, sans se déplacer vers des points de rassemblement, sans subir les cris, les abus ou les attentes interminables. Il redonne au voyageur le contrôle total sur son déplacement et élimine les obstacles inutiles qui rendent les trajets pénibles.

\section{Sécurité, confiance et responsabilité}

La sécurité et la confiance sont au cœur du service. Les conducteurs doivent vérifier leur permis et les documents du véhicule, tandis que les voyageurs peuvent vérifier leur identité. Les données sont cryptées et ne sont utilisées qu'en cas d'incident ou de litige. Un badge indique si un profil est vérifié, et un historique simple permet d'évaluer le sérieux d'un utilisateur. L'objectif n'est pas de créer une surveillance intrusive, mais de garantir un minimum de transparence et de fiabilité pour protéger les usagers.

\section{Une gouvernance communautaire}

La gouvernance du service repose sur une modération communautaire. Des volontaires issus de la communauté traitent les signalements, gèrent les litiges simples et veillent au respect des règles. Ils agissent de bonne foi, en toute neutralité, et transmettent les cas sérieux aux administrateurs. Les administrateurs, quant à eux, assurent la sécurité générale du service, prennent les décisions finales concernant les suspensions ou retraits, et garantissent la transparence du fonctionnement. Ils veillent également à la protection des données et à l'application des politiques définies.

\section{Conclusion}

Ce chapitre présente une vision claire : rendre la mobilité plus simple, plus juste et plus accessible, en s'appuyant sur une solution souveraine, gratuite et pensée pour les réalités africaines. \ngelmak ne cherche pas à exploiter un marché, mais à résoudre un problème. L'objectif est d'offrir un service qui fonctionne, qui aide, qui soulage, et qui démontre que l'Afrique peut construire ses propres outils, adaptés à ses besoins et à ses ambitions. En réorganisant un système déjà existant mais chaotique, en supprimant les intermédiaires abusifs et en redonnant le contrôle aux usagers, nous posons les bases d'une mobilité plus humaine, plus efficace et plus respectueuse de chacun.